% -*- latex -*-

The evolution of the computing world from teraflop to petaflop has been
relatively effortless, with several of the existing programming models
scaling effectively to the petascale. The migration to exascale, however,
poses considerable challenges. All industry trends infer that the exascale
machine will be built using processors containing hundreds to thousands of
cores per chip. It can be inferred that efficient concurrency on exascale
machines requires a massive amount of concurrent threads, each performing
many operations on a localized piece of data.

Currently, visualization libraries and applications are based off what is
known as the visualization pipeline. In the pipeline model, algorithms are
encapsulated as \keyterm{filters} with inputs and outputs. These filters
are connected by setting the output of one component to the input of
another. Parallelism in the visualization pipeline is achieved by
replicating the pipeline for each processing thread. This works well for
today's distributed memory parallel computers but cannot be sustained when
operating on processors with thousands of cores.

Our project investigates a new visualization framework designed to exhibit
the pervasive parallelism necessary for extreme scale machines. Our
framework achieves this by defining algorithms in terms of
\keyterm{worklets}\index{worklet}, which are localized stateless
operations. Worklets are atomic operations that execute when invoked unlike
filters, which execute when a pipeline request occurs. The worklet design
allows execution on a massive amount of lightweight threads with minimal
overhead. Only with such fine-grained parallelism can we hope to fill the
billions of threads we expect will be necessary for efficient computation
on an exascale machine.


\section*{Progress and Accomplishments}
\addcontentsline{toc}{section}{Progress and Accomplishments}

Although the ``Pervasive Parallel Processing Framework for Data
Visualization and Analysis at Extreme Scale'' project is a research project
to make progress on designing and implementing massively threaded
visualization algorithms, our project also aims to explore techniques that
simplify the development of such algorithms and to provide useful software
for this purpose. To that end we have developed the Dax toolkit as a
deployment platform for our research. The Dax toolkit is a comprehensive
C++ header library that embodies the techniques developed within the
project. A summary of our major accomplishments is as follows.

\paragraph*{Development of Framework}

Much thought has gone into the design of the core components of the toolkit
API that users will use to define their analysis algorithms. The API has
been developed to be succinct, type safe, and efficient when executing on
multiple architectures. We developed adapters to enable execution and
testing of the framework on multiple architectures including GPUs and
CPUs. We added support for advanced core data structures needed for data
analysis such as data set types, cell types, and structures for sorting
geometrical and topological information. The framework supports advanced
analysis algorithms including those that change both geometry and topology,
such as marching cubes and threshold.
  
We reevaluated the use of pipelines for analysis on massively parallel
architectures. The idea being that the framework would control the flow of
execution and perform a limited sort of kernel fusion to maximize the
amount of computation per data load. We concluded that the complexity in
API and the toolkit implementation due to scheduling and execution of
pipelines could be greatly simplified by abandoning the connected pipeline
paradigm. Instead, the users directly manage the data flow by dispatching
calls to analysis worklets in order. We find that the framework structure
lends to developing in such a way as to encourage performing as many
operations per data load as possible without further adjustment by the
framework.

\paragraph*{Cross-Platform Analysis}

Our implementation now supports multiple target platforms including GPU
(using CUDA\lcite{CUDA} via Thrust\lcite{Thrust}), multi-core CPUs (using
either OpenMP\lcite{OpenMP} via Thrust or TBB\lcite{TBB}), and single core
CPU. We achieve these cross-platform implementations through careful
structure of the toolkit code. We have identified the basic features
specific to each multithreaded device and encapsulated them in a unit
called a \index{device~adapter}\keyterm{device adapter}. A device adapter
can be implemented by providing only a thread scheduling mechanism although
more efficient custom algorithms can be provided as well. A device adapter
can be changed with a single template parameter, thus enabling the porting
of the majority of code with very little development.

\paragraph*{Development of Analysis Algorithms}

We developed infrastructure to support analysis algorithms including those
that change topology and geometry. These algorithms require multiple
passes, particularly on architectures with memory restrictions, such as
GPUs and potentially proposed exascale machines. With the framework API and
infrastructure matured we developed key analysis algorithms that also
exercise the framework. These include thresholding of cells using point
fields, marching cubes for generating isosurfaces, and computing derived
fields.

\paragraph*{Software Infrastructure}

For software reliability and correctness, we set up a software process
including a testing framework, dashboards for daily regression testing and
verification, a developer wiki for design and implementation discussions,
doxygen for API documentation, and mailing list for developer
communication.


\section*{Presentations and Publications}
\addcontentsline{toc}{section}{Presentations and Publications}

During the course of our project, we presented our work to a broad audience
to gain feedback and discuss the vision we have for parallelizing
visualization algorithms. Some of the major locations where we presented
the Dax toolkit are:

\begin{description}
  \raggedright
\item\textbf{Dax: Data Analysis at Extreme,} paper by Kenneth Moreland,
  Utkarsh Ayachit, Berk Geveci, and Kwan-Liu Ma. In Proceedings of SciDAC
  2011, July 2011.
\item\textbf{Dax Toolkit: A Proposed Framework for Data Analysis and
  Visualization at Extreme Scale,} paper by Kenneth Moreland, Utkarsh
  Ayachit, Berk Geveci, and Kwan-Liu Ma. In IEEE Symposium on Large-Scale
  Data Analysis and Visualization (LDAV), October 2011.
\item\textbf{Next-Generation Capabilities for Large-Scale Scientific
  Visualization,} presentation by Kenneth Moreland. 15th SIAN Conference on
  Parallel Processing for Scientific Computing, February 2012.
\item\textbf{Next-Generation Codes/Portability: Dax Perspective,}
  presentation by Kenneth Moreland, DOECGF, April 2012.
\item\textbf{Oh, \$\#*@! Exascale! The Effect of Emerging Architectures on
  Scientific Discovery,} paper by Kenneth Moreland. In 2012 SC Companion
  (Proceedings of the Ultrascale Visualization Workshop), November 2012,
  pg. 224-231. DOI 10.1109/SC.Companion.2012.38.
\item\textbf{Dax for Multi- and Many-Core Architectures,} panel
  presentation by Kenneth Moreland. Supercomputing, November 2012.
\item\textbf{The SDAV Software Frameworks for Visualization and Analysis on
  Next-Generation Multi-Core and Many-Core Architectures,} paper by
  Christopher Sewell, Jeremy Meredith, Kenneth Moreland, Tom Peterka, Dave
  DeMarle, La-ta Lo, James Ahrens, Robert Maynard, and Berk Geveci. In 2012
  SC Companion (Proceedings of the Ultrascale Visualization Workshop),
  November 2012, pg. 206-214. DOI 10.1109/SC.Companion.2012.36.
\item\textbf{Flexible Analysis Software for Emerging Architectures,} paper
  by Kenneth Moreland, Brad King, Robert Maynard, and Kwan-Liu Ma. In 2012
  SC Companion (Proceedings of Petascale Data Analytics: Challenges and
  Opportunities), November 2012. DOI 10.1109/SC.Companion.2012.115.
\item\textbf{Optimizing Threshold for Extreme Scale Analysis,} poster by
  Robert Maynard, Kenneth Moreland, Utkarsh Ayachit, Berk Geveci, and
  Kwan-Liu Ma. In Proceedings of SPIE Visualization and Data Analysis,
  February 2013.
\item\textbf{A Survey of Visualization Pipelines,} paper by Kenneth
  Moreland. IEEE Transactions on Visualization and Computer Graphics,
  19(3), March 2013. DOI 10.1109/TVCG.2012.133.
\item\textbf{DaxToolkit: Efficient Visualization at Extreme Scale,}
  presentation by Robert Maynard, GPU Technology Conference, March 2013.
\item\textbf{The effect of emerging architectures on data analysis
  software,} panel presentation by Kenneth Moreland, SOS 17, March 2013.
\item\textbf{Dax,} presentation by Kenneth Moreland, DOECGF, April 2013.
\item\textbf{Research Challenges for Visualization Software,} paper by Hank
  Childs, Berk Geveci, Will Schroeder, Jeremy Meredith, Kenneth Moreland,
  Christopher Sewell, Torsten Kuhlen, and E. Wes Bethel. IEEE Computer,
  46(5), May 2013, pg. 34-42. DOI 10.1109/MC.2013.179.
\item\textbf{Upcoming Challenges for Scientific Visualization Software:
  Programming Future Architectures,} panel presentation by Kenneth
  Moreland, IEEE Visualization, October 2013.
\item\textbf{A Classification of Scientific Visualization Algorithms for
  Massive Threading,} paper by Kenneth Moreland, Berk Geveci, Kwan-Liu Ma,
  and Robert Maynard. In Proceedings of the Ultrascale Visualization
  Workshop, November 2013. DOI 10.1145/2535571.2535591.
\end{description}
