% -*- latex -*-

High-performance computing relies on ever finer threading. Recent advances
in processor technology include greater numbers of cores, hyperthreading,
and accelerators with integrated blocks of cores, all of which require more
software parallelism to achieve peak performance. Current visualization
solutions cannot support this extreme level of concurrency. Extreme scale
systems require a new programming model and a fundamental change in how we
design algorithms. This document is a report on the project titled ``A
Pervasive Parallel Processing Framework for Data Visualization and Analysis
at Extreme Scale'' funded by the ASCR Scientific Data Management and
Analysis at Extreme Scale program. This project delivers its work with the
creation of the Data Analysis at Extreme (Dax) toolkit.

The Dax toolkit supports a number of algorithms and the ability to design
further algorithms through a top-down design with an emphasis on extreme
parallelism. Dax also provides support for finding and building links
across topologies, making it possible to perform operations that determine
manifold surfaces, interpolate generated values, and find
adjacencies. Although Dax provides a simplified high-level interface for
programming, its template-based code removes the overhead of abstraction.

\begin{figure}
\centering
\begin{tabular}{cc}
\begin{lstlisting}
int vtkCellDerivatives::RequestData(...)
{
  ...[allocate output arrays]...
  ...[validate inputs]...




  for (cellId=0;
       cellId < numCells;
       cellId++)
    {
    ...[update progress]...
    input->GetCell(cellId, cell);
    inScalars->GetTuples(cell->PointIds,
                         cellScalars);
    scalars = cellScalars->GetPointer(0);

    subId =
      cell->GetParametricCenter(pcoords);


    cell->Derivatives(
      subId, pcoords, scalars, 1, derivs);
 
    outGradients->SetTuple(cellId,derivs);

    }
  ...[cleanup]...
}
\end{lstlisting}
&
\begin{lstlisting}
struct CellGradient
  : public dax::exec::WorkletMapCell
{
  typedef void ControlSignature(
      Topology, Field(Point),
      Field(Point), Field(Out));
  typedef _4 ExecutionSignature(_1,_2,_3);
 
  template<class CellTag>
  DAX_EXEC_EXPORT
  dax::Vector3 operator()(...)
  {






    dax::Vector3 parametricCellCenter =
        dax::exec::ParametricCoordinates<
          CellTag>::Center();
 
    return dax::exec::CellDerivative(
        parametricCellCenter,
        coords,
        pointField,
        cellTag);
  }
 
};
\end{lstlisting}
\\
VTK Code & Dax Code
\end{tabular}
\caption[Comparison of VTK code and Dax code.]{A comparison of code to
  compute a localized field derivative within VTK and within the Dax
  toolkit.}
\label{fig:CompareVTKDax}
\end{figure}

The Dax toolkit simplifies the development of parallel visualization
algorithms. Consider the code samples in Figure~\ref{fig:CompareVTKDax}
that come from the Visualization Toolkit (VTK) on the left and our Dax
toolkit on the right. Both implementations perform the same operation; they
estimate gradients using finite differences. Both toolkits provide similar
classes and functions, and consequently the code looks remarkably similar.

However, because the Dax toolkit is structured such that it can schedule
its execution on a GPU, we measure that it performs this operation over 100
times faster than the VTK code running on a single CPU. Furthermore, the
Dax API can be switched to a different device by changing only a single
line of code. Dax currently provides scheduling for CUDA (GPU), OpenMP
(multi-core CPU), Intel Threading Building Blocks (multi-core CPU), and
serial execution.

This report documents the design of the Dax
toolkit. Chapter~\ref{chap:Overview} provides an overview of the Dax
toolkit and describes the higher level
context. Chapter~\ref{chap:Documentation} describes the API of the Dax
toolkit and provides the initial software
documentation. Chapter~\ref{chap:ProgressReport} reports on further lessons
and achievements attained during this project.
